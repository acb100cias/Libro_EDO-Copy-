
\section{Una idea intuitiva}\index{Una idea intuitiva}

Supongamos que tenemos los siguientes datos de una población de bacterias:
%%Datos
La gráfica de los datos luce de la siguiente manera:
%%Gráfica
Ahora, notemos que podemos calcular, dado los datos, la pendiente entre cualesquiera
dos puntos. La pendiente es:
\begin{equation}
    m_{i} = \displaystyle\frac{\Delta y_{i}}{\Delta t_{i}} = \frac{y_{i+1}-y_{i}}{t_{i+1}-t_{i}}
\end{equation}


Ahora, de alguna forma lo que esto nos quiere decir es que existe una curva $x(t)$ que evaluada en el intervalo $\delta t_{i}$ tiene una pendiente $x'(\Delta t_{i}) = m_{i} $. 
\ \\
\noindent
En realidad hay muchas curvas que cumplen que $x'(\Delta t_{i}) = m_{i}$ sin embargo existe sólo una curva que además de eso $x(t_{i}) = p_{i}$. Notemos que además esta relación define una relación biyectiva entre los datos del tiempo y el tamaño de la población, podemos aventurarnos a decir que el dominio de la curva $x(t)$ es justamente los el conjunto $T = \left\{ t : t \in [0,15]\subset \mathbb{R}\right\}$ y la imagen son todos los puntos correspondientes a la poblaci�n. 
\ '\\
\noindent
Ahora bien, consideremos los datos que se reportan de otro experimento