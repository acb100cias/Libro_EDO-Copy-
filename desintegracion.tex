\section{Decaimiento radioactivo}

La velocidad  de la desintegración radioactiva es proporcional a la cantidad $x$ de sustancia no desintegrada. Supongamos que en un momento inicial $t_0$ tenemos una masa $x(t_0)$ y $k$ es el coeficiente de desintegración, el fenómeno queda descrito por el siguiente problema de valores iniciales

\begin{eqnarray}
\dot{x}&=&-kx\\
x(t_0)&=&x_0
\end{eqnarray}

La solución para este problema de valores iniciales ya la conocemos
$$
x(t)=x_0e^{-k(t-t_0)}
$$

Ahora qué tanta información obtenemos del conocimiento de esta solución. Podemos calcular el tiempo en el cual ya se ha desintegrado la mitad de la masa inicial. Haciendo $\tau=t-t_0$ tenemos:
$$
\frac{x_0}{2}=x_0e^{-k\tau}
$$
 así, tomando $x_0 =1$
 $$
 \frac{1}{2}=e^{-k\tau}
 $$
  y finalmente
  $$
  \ln(\frac{1}{2})=-k\tau
  $$
  o equivalentemente
  $$
  \tau=\frac{\ln 2}{k}
  $$
  
  