\section{Procesos a tasa constante}

En la Naturaleza existe una gran cantidad de procesos que a lo largo del tiempo cambian o evolucionan a {\it tasa constante}. En esta sección haremos las definiciones esenciales y en las próximas veremos ejemplos de ellos.

Sin embargo, primero queremos precisar que cuando hablemos de {\it Naturaleza } entenderemos que comprende al mundo vivo --humano o no humano-- y los fenómenos que involucran a la materia inerte.

Vamos a  comenzar por lo básico. Supongamos que tenemos una magnitud $p(t)$ que es función del tiempo. Dicha magnitud puede representar a cualquier variable de interés de estudio: tamaño de una población, cantidad de dinero, número de enfermos, cantidad de material radioactivo, etcétera. Empezaremos por suponer que $p(t)$ solamente puede evaluarse en valores discretos del tiempo por lo cual es de interés estudiar la sucesión:
\[
\{p(t)\}_{t=0}^n = p(0),p(1),p(2),\hdots p(n)
\]
\noindent donde $n$ es un valor arbitrario del tiempo.


\[
 \dfrac {p(n+1) -p(n) }{p(n)}=q\\
 \]

