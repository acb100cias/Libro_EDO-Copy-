\section{Un modelo de poblaciones más realista}

En la primera sección vimos que el modelo propuesto por Malthus  además de ser políticamente malintencionado, es a todas luces equivocado, ninguna población puede crecer permanente e ilimitadamente. Cualquier población ve acotado su crecimiento por cierta ``presión'' ejercida por el ambiente. ¿Cómo modelamos esta ``presión''? El problema es ahora proponer un modelo para una población cuyo crecimiento quede acotado por su propia dinámica. Llamemos $K$ a tal presión ambiental, en la literatura se le llama capacidad de carga del sistema, representa la cantidad de individuos de cierta población, a los que el medio puede proveer de subsistencia. Nuevamente consideraremos que la tasa intrínseca (natural) de crecimiento de la población es $\rho$ y que el crecimiento es proporcional no sólo al tamaño de la población actual, sino que decrece proporcionalmente a la proporción entre la población actual y la capacidad de carga.

\begin{equation}
    \dot{x}=\rho(1-\frac{x}{K})x
\end{equation}

Lo primero que debemos notar es que esta ecuación ya no es lineal en $x$, tiene un término cuadrático que de alguna forma nos da información sobre las interacciones intraespecíficas de la población. Ahora busquemos las soluciones de equilibrio, es decir aquellas para las cuales no hay cambios en la población. En este caso es claro que cuando $x=0$ ó $\left(1-\dfrac{x}{K}\right)=0$, entonces $\dot{x}\equiv 0$. Al caso  $x=0$ lo llamaremos el equilibrio trivial, y corresponde a que no hay ningún cambio en la población, cuando   $\left(1-\dfrac{x}{K}\right)=0$ tenemos que $x\equiv K$, es decir que al alcanzar la capacidad de carga del sistema el crecimiento se detiene. 

Consideremos la función polinomial $x\mapsto \rho(1-\frac{x}{K})x$ es una cuadrática cóncava en la dirección negativa del eje $\mathrm{Y}$, que cruza al eje horizontal en los puntos $(0,0)$ y $(0,K)$ el punto máximo de la parábola $\rho x-\rho \dfrac{x^2}{K}=0$ se alcanza cuando $\rho-2\rho \dfrac{x}{K}=0$ o equivalentemente $x=\dfrac{K}{2}$, así para distintos valores de $K$, el máximo de la parábola está más o menos alejado de la gráfica de la identidad. Las soluciones de la ecuación se alejan del $(0,0)$ y se aproximan a $(0,K)$. 

Nótese que si la condición inicial es menor que el parametro $K$, entonces . Ahora si iniciamos en una condición $x_0 > K$ entonces el cociente $\dfrac{x_0}{K}>1$ y la derivada $\dot{x} < 0$ y por lo tanto las soluciones son decrecientes y en ambos casos se aproximan asintóticamente a $x\equiv K$, en efecto tomamos el límite

Ahora analicemos el caso más sencillo. Tomamos $K=1$. Si aplicamos el truco de la separación de variables, tenemos la ecuación integral

\begin{equation}
    \int \frac{\mathrm{d}x}{x(1-x)}=\rho\int\mathrm{d}t
\end{equation}
resolviendo por fracciones parciales tenemos
$$
\int\frac{\mathrm{d}x}{x}+\int\frac{\mathrm{d}x}{1-x}=\rho x +\xi_0
$$
tenemos que:
$$
\ln(\frac{x}{(1-x)}=\rho t+\xi_1
$$
se sigue que:
$$
x(t)=\xi\frac{e^{\rho t}}{1+e}
$$